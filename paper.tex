% 300 Paper, W '13

% Two Column Format
\documentclass[11pt]{article}
%this allows us to specify sections to be single or multi column so that things like title page and table of contents are single column
\usepackage{multicol}
\usepackage{verbatim}

\usepackage{setspace}
\usepackage{url}

%%% PAGE DIMENSIONS
\usepackage{geometry} % to change the page dimensions
\geometry{letterpaper}

\setcounter{secnumdepth}{5}


\begin{document}

%%%%%%%%%%%%%%%%%%
%%% Cover Page %%%
%%%%%%%%%%%%%%%%%%

\title{\vfill World of Warcraft's New Way to Get You Hooked} %\vfill gives us the black space at the top of the page
\author{
By Taggart Ashby \vspace{10pt} \\
CSC 300: Professional Responsibilities  \vspace{10pt} \\
}
\date{\today} %Or use \Today for today's Date

\maketitle

\vfill  %in combination with \newpage this forces the abstract to the bottom of the page
\thispagestyle{empty} %remove page number from title page
\newpage

%end the 1 column format


%start 2 column format
\begin{multicols}{2}
%Start numbering first page of content as page 1
\setcounter{page}{1}

\section{Focus Question}
Is Blizzard's World of Warcraft ethical with its perceived ability to make users addicted?

%%%%%%%%%%%%%%%%
%%% Analysis %%%
%%%%%%%%%%%%%%%%

\section{Analysis}

To aid in my analysis of whether or not Blizzard Entertainment's World of Warcraft trial model is ethical, the Software Engineering Code of Ethics (SE Code) will be used. The SE Code applies to ``professional software engineers, including practitioners, educators, managers, supervisors and policy makers, as well as trainees and students of the profession.'' \cite{SECode} Blizzard Entertainment's World of Warcraft is a software system that runs on computers. \cite{WoWSystemReqs} Software systems are written by software engineers. \cite{ACMApprovesSECode} According to the Computer Society and the Association for Computing Machinery (ACM), all professional software engineers must abide by the rules outlined in the the SE Code. \cite{ACMApprovesSECode}\cite{SECode} Blizzard Entertainment can be seen as a ``practitioner'' in that they are actively engaged in making software, in addition to some of its employees who are software engineers. Therefore, Blizzard Entertainment and Blizzard Entertainment's software engineers are subject to the SE Code. \cite{SECode} Being subject to the SE Code allows me to use it to conclude the ethics of World of Warcraft's trial model.

\subsection{Definition of ``addiction''}
Because this paper will often reference psychological addiction, it is important to make sure this notion is concrete before continuing.\\
A person who is addicted to something will pathologically pursue rewards and/or relief through substance abuse or other behaviors, in this case playing World of Warcraft. \cite{DefinitionOfAddiction}

Addiction can be seen as an \emph{inability to abstain} from something, \emph{impairment in behavior control} regarding this thing, and a \emph{craving} for this thing. There will be a diminished recognition of issues with one's behavior, suffering of interpersonal relationships, and a dysfunctional emotional response. \cite{DefinitionOfAddiction}

\subsection{SE Code Section 1.02\\Moderate the interests of the software engineer, the employer, the client and the users with the public good.}
In the next few sections this tenet is broken down into its parts. I will explore the ``interests of the software engineer,'' in this case World of Warcraft's software engineers, the ``interests of ... the employer,'' in this case Blizzard Entertainment since it employs the software engineers, and lastly the ``interests of ... the client and the users,'' the players who use World of Warcraft. Following that will be a discussion of the ``public good.''\\
After substituting the above into the rule, SE Code Section 1.02 becomes: ``World of Warcraft's Software Engineers shall moderate the interests of themselves, Blizzard Entertainment, and the players of World of Warcraft with the public good.''

\subsubsection{Interests of the Software Engineer}
It is common sense that a software engineer strives to make working software. For World of Warcraft's software engineers, this success is measured by how many players enjoy new content and continue to play the game. Considerations are made for balance and gameplay as well as story. \cite{DevWatercooler} These engineers are also interested in keeping their jobs and continuing to make money. In order to keep their jobs, their interests must lie in getting players buy the software and then to come back to the game world often and play the game.

\subsubsection{Interests of Blizzard Entertainment}
Blizzard Entertainment is a publicly traded company. \cite{BlizzStock} As such its main goal is to show its shareholders that it is profiting and increasing the volume of business it handles. In no uncertain terms, Blizzard Entertainment wants as many people to buy and play their games, World of Warcraft included, as possible. Since World of Warcraft is subscription-based \cite{WoWSubscription}, Blizzard Entertainment earns money every single month from World of Warcraft players. None of the other games that Blizzard makes are subscription based. \cite{BlizzGameList} That means that Blizzard only ever receives one payment for any of its other games from customers. Blizzard Entertainment's interests are in getting more players to buy their games, particularly World of Warcraft which nets them \$13-15 every month in addition to the initial cost. \cite{WoWSubPlans}

\subsubsection{Interests of the Player}
By purchasing a game like World of Warcraft, players are committing themselves to pay Blizzard Entertainment every month. \cite{WoWSubscription} Some players are reaping benefits that they would not be able to get otherwise, such as an increase in social interactions. \cite{IsThereEvidenceOfInternetAddiction} Users who play the game rarely navigate the world alone, instead they form groups with other players, join guilds, and tackle quests and dungeons that require teamwork and talking to others.
One particular study found that MMORPG users play for a mixture of ten reasons: advancement, mechanics, competition, socializing, relationship, teamwork, discovery, role-playing, customization, and escapism. These ten reasons break down into three components: achievement, social, and immersion. \cite{PlayerMotivations} The interests of the player boil down to a sense of achievement, having social interactions, and immersion in a world outside their own.

\subsubsection{``Public Good'' Breakdown}
In 1996 the U.S. Supreme Court ruled that ``the mental health of the Nation's citizenry ... is a \emph{public good} of transcendent importance.'' \cite{SupremeCourtPublicGood} Addiction is a mental disease \cite{DefinitionOfAddiction} and therefore goes against the public good. A healthy population is not addicted to World of Warcraft and therefore exhibits the ability to abstain from World of Warcraft, have unimpaired behavior control, and not crave playing World of Warcraft.

First will be a purely numerical, Utilitarian approach to public good, `` ``Public Good'' By The Numbers''. The metric used will be the percentage of the population addicted to World of Warcraft. I find this argument to be a bit heartless and cold and so I will follow it with a section that shows that the public good is not only upheld, but improved, by World of Warcraft, `` ``Public Good'' Being Improved''. These improvements come from World of Warcraft helping some users overcome deficiencies in themselves, offering players powerful emotion experiences, and providing new or improved life skills.

\subsubsection{``Public Good'' By The Numbers}
In order to argue that the public good is being upheld I assert that Utilitarianism is the most effective ethical system to use because it judges actions based on the net happiness of everyone. Utilitarianism is ``[t]he creed which ... holds that actions are right in proportion as they tend to promote happiness, wrong as they tend to produce the reverse of happiness.'' \cite{Utilitarianism}

The most shocking statistic I found came from Dr. Maressa Orzack, founder and coordinator of Computer Addiction Services \cite{CompAddictionServices}, who cites that up to 40\% of World of Warcraft players could be addicted. \cite{FortyPercentAddicted} If everyone in the world played World of Warcraft, a total of over 7 billion people \cite{WorldPopulation}, and Dr. Ozark's prediction is correct, 2.8 billion people would be addicted. Based purely on the numbers that leaves 60\% of the world's population that does not have the mental disease. A utilitarian view in this case would be that World of Warcraft does not create mental disease in the majority of the world and therefore does not diminish mental or physical welfare of the public.

\subsubsection{``Public Good'' Being Improved}
Simple statistics and numbers are not enough to convince me that World of Warcraft is doing good by its players. World of Warcraft has over 10 million subscribers \cite{WoWPlayerCount} who average 22 hours of play time per week. \cite{PlayerMotivations} A study published in \emph{Cyberpsychology \& Behavior} \cite{ExcessiveUseForSocialAspects} shows that although World of Warcraft players are passionate about their experiences in the game, they do not show signs of addiction.

\paragraph{Users Counteract Disabilities and Deficiencies\\}
In a case study of five ``problematic'' internet and game users, users who are seen by family and friends as having extreme computer addiction issues, only two showed signs of true addiction while the remaining three seemed to be using the internet to counter-act other deficiencies. \cite{IsThereEvidenceOfInternetAddiction} These three users who were not psychologically addicted used the internet and games to connect with others in ways they couldn't previously. Factors keeping them from doing so prior were due to physical appearance, social anxiety, and mental disability. One European man found a wife in the United States and moved to be with her, improving both their lives, thanks to his extended internet use. \cite{IsThereEvidenceOfInternetAddiction} These people use the internet and games as a way to enrich their lives, not detract from them.

\paragraph{Blizzard Gets Thank-You Letters\\}
During a panel at D.I.C.E. 2012 \cite{DICEInterview} that included Mike Epps, President of Epic Games, Ted Price, CEO and Founder of Insomniac Games, and Frank Pearce, current VP of product development and one of the three founders of Blizzard, Frank Pearce discussed a letter he received from a fan who had a disability and found comfort in World of Warcraft. He went on to say that Blizzard Entertainment is doing so many people a service that they actually receive letters thanking them for making such an immersive experience. \cite{DICEInterview}

\paragraph{World Of Warcraft Teaches Players Life Skills\\}
In a three year study of over 30,000 MMORPG users, over 50\% of users said that they learned valuable leadership skills that have helped them outside of the game. \cite{MotivationsAndDerviedExperiences}
Many features in World of Warcraft drive the players to socialize. \footnote{Having over two months of play time in-game with the last play session less than a week ago, I consider myself qualified to discuss the mechanics of World of Warcraft without citation.} The best gear and most prestigious titles are only available to users who group together to take on challenges that cannot be conquered by oneself. These challenges require intense coordination and communication as well as skilled players. Without grouping together, players have no chance of getting these items or attaining these statuses.
Guilds offer perks to their members through mechanics built into the game. As guilds grow in size and their members complete quests and dungeons, these guilds level up and offer a variety of rewards. Becoming part of a community of people is rewarded in-game.
\newline
\newline
Friendships and powerful experiences help us grow as individuals. World of Warcraft is improving the public good by improving the individuals that play it in ways that promote social skills and leadership skills. These skills benefit not only the player but also the people the player interacts with outside of the game. MMORPGs, like World of Warcraft, are providing their users with leadership skills, strong friendships, and powerful experiences that they take with them to interactions outside the game.

\subsubsection{Ethical Conclusion}
In order for Blizzard's actions to be ethical, its engineers must ``moderate the interests of themselves, Blizzard Entertainment, and the players of World of Warcraft with the public good.''

Players are interested in achievement, social interactions, and immersion. Blizzard's engineers are interested in creating those exact experiences and content. The interests of the software engineer, creating content that players love, and the interests of the player are complimentary. In this way, the engineers are moderating their interests with the players.

Blizzard Entertainment's interests of growing as a company and keeping its shareholders satisfied are only improved by happy players. Happy players will recommend Blizzard and its products to friends as well as continue to pay monthly fees for World of Warcraft and buy Blizzard's other games. This is how Blizzard's and the player's interests are moderated.

Having a majority of players who are \emph{not} addicted, and thus playing for pleasure or other purposes, is in line with the Utilitarian view of ethical as World of Warcraft is not promoting mental disease and therefore in line with a cold, calculated numerical view of ``public good.''

Not only are a majority of users not addicted, over 50\% of these users are cultivating life skills, strong friendships, and powerful emotional experiences. \cite{MotivationsAndDerviedExperiences} These users are growing as individuals and going out into the world as stronger people with leadership and teamwork skills. Because the game is improving users, World of Warcraft, and other MMORPGs, can be seen as \emph{improving} the public good. By improving the public good, Blizzard's engineers are moderating the interests of all parties with the public good.

Blizzard's software engineers are moderating all parties' interests. The public good is not only being upheld, but improved. Therefore, Blizzard Entertainment's actions are not only ethical in accordance with SE Code section 1.02, but can be seen as extraordinary, going beyond moderating their interests and improving the public good.

\end{multicols}
\newpage

\bibliographystyle{IEEEannot}

\bibliography{paper}

\end{document}
